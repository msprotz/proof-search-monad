\documentclass[final,xetex]{beamer}

\input{talks/common/macros.tex}
\input{papers/common/macros.tex}
\input{bubbles.tex}

% ------------------------------------------------------------------------------

\usepackage{ucharclasses}
  \setTransitionsFor{Dingbats}{\begingroup\fontspec{DejaVu Sans}[Scale=MatchLowercase]}{\endgroup}
\usepackage{multirow}
\usepackage{mathtools}
\usepackage[normalem]{ulem}

%\def\mintedusecache{}
\def\mintedwritecache{}

% ------------------------------------------------------------------------------

\newcommand{\sidenote}[1]{
\raggedleft
\footnotesize
\itshape
#1
}

\newcommand{\statement}[1]{
  \begin{frame}[plain]
  \begin{center}
    \Large
    \textcolor{skyblue3}{
    #1
    }
  \end{center}
  \end{frame}
}

\newcommand{\V}{\ensuremath{\mathcal{V}}} % The prefix
\newcommand{\R}{\ensuremath{\mathcal{R}}} % Rigid
\newcommand{\F}{\ensuremath{\mathcal{F}}} % Flexible

% ------------------------------------------------------------------------------

\TangoColors
\mode<presentation> {
  \setbeamertemplate{navigation symbols}{}
  \setbeamertemplate{items}[circle]
  \setbeamercolor*{Title bar}{bg=white,fg=chameleon3}
  \setbeamersize{text margin left=2em}
  \setbeamertemplate{footline}{
    \leavevmode%
    \hbox{%
    \begin{beamercolorbox}[wd=.3\paperwidth,ht=2.25ex,dp=1ex,center]{author in head/foot}%
      \usebeamerfont{author in head/foot}\insertshortauthor\ —\ \insertshortinstitute
    \end{beamercolorbox}%
    \begin{beamercolorbox}[wd=.4\paperwidth,ht=2.25ex,dp=1ex,center]{title in head/foot}%
      \usebeamerfont{title in head/foot}\insertshorttitle
    \end{beamercolorbox}%
    \begin{beamercolorbox}[wd=.3\paperwidth,ht=2.25ex,dp=1ex,right]{date in head/foot}%
      \usebeamerfont{date in head/foot}\insertshortdate{}\hspace*{2em}
      \insertframenumber{} / \inserttotalframenumber\hspace*{2ex}
    \end{beamercolorbox}}%
    \vskip0pt%
  }
}

\title[The Proof Search Monad]{
  Functional Pearl: the Proof Search Monad
}
\author[J. Protzenko]{
  \newauthor{Jonathan Protzenko}{protz@microsoft.com}
}

\institute{Microsoft Research}
\date{Nov. 23th, 2015}

\begin{document}

\begin{frame}
  \begin{center}
    {\raggedright\large (functional pearl)\\
    \Huge\vspace{.2em}
    \hspace{1.2em}\green{\emph{the Proof Search Monad}}}
  \end{center}

  \vspace{3cm}

    Jonathan Protzenko (Microsoft Research)\\
    \scriptsize
    \texttt{protz@microsoft.com}
\end{frame}

\begin{frame}{The Proof Search Monad}
  A \green{library} developed to implement the type-checker of \mezzo.

  \bigskip

  In this talk:
  \begin{itemize}
    \item what is \mezzo (from a very high-level)
    \item generalize (when is this library suitable)
    \item the library itself (combinators and implementation).
  \end{itemize}
\end{frame}

\begin{frame}[plain]
  \backupslide
  \MzTitle{Where it all came from}
\end{frame}

\begin{frame}{My thesis in one slide}
  ``Separation logic as a type system''. % who's familiar with it?

  \begin{itemize}
    \item \green{\mezzo}: \emph{barely} a type system
      {\tiny(flow-sensitive, structural information, keeps track of local
      aliasing)}
    \item Hindley-Milner: \green{nope}
    \item Because: \green{undecidable} {\tiny(System F, entailment, framing,
      higher-order logic)}
    \item But: we \emph{still} want \green{inference}
    \item So: \emph{\green{heuristics}} (it ``mostly'' works)
  \end{itemize}
\end{frame}

\begin{frame}[fragile]{Flow-sensitivity}

  \mezzo has \green{permissions}, of the form $\tyatomic xt$, separated by
  $\ast$.

  \bigskip

  \begin{description}[\hspace{10em}]
    \item[In ML:] $\Gamma = x: t, y: u$
    \item[In \mezzo:] $P = \tyatomic xt \ast \tyatomic yu$
  \end{description}

  \bigskip

  \bigskip

  \MzStartBubbles
\begin{mzcode}
val f (x: ...): ... = #\MzProgramPoint#
  let y = ... in #\MzProgramPoint#
  ...#\MzProgramPoint#
\end{mzcode}

  \MzStartBubbles
  \MzAutoBubble{2em}{2em}{1cm}{}{%
    P₁
  }
  \MzAutoBubble{0em}{3em}{1cm}{}{%
    P₂
  }
  \MzAutoBubble{2em}{-2em}{1cm}{}{%
    P₃
  }

  \bigskip

  \bigskip

  \only<5>{
  This allows keeping track of \green{ownership}.
  }

\end{frame}

\begin{frame}[fragile]{Why ownership?}
  At each program point, the programmer knows which objects they own, and how
  they own them.
  Ownership is either shared (duplicable) or unique (exclusive).

  \vfill

  \begin{description}
    \item[Data-race freedom]
      \emph{Mutable} objects have a unique owner. Therefore, at most one thread
      may mutate an object at any given time. \mezzo programs are data-race
      free.
    \item[State change]
      If I'm the sole owner of an object, I'm allowed to break its invariants
      (type). Important because of fine-grained aliasing tracking.
  \end{description}
\end{frame}

\begin{frame}[fragile]{Ownership example}

  Data-race \green{freedom} \emph{and} \green{state change}.

  \bigskip

  \bigskip

  \begin{mzcode}
let r = newref x in
(* r @ Ref { contents = x } *)
r := y;
(* r @ Ref { contents = y } *)
  \end{mzcode}

\end{frame}

\begin{frame}{A rich type system\ldots}
  \begin{description}[\hspace{10.2em}]
    \item[Singleton types] \code{x @ (=y)}: \code{x} \green{is} \code{y}
      \\ Written as: $x = y$
    \item[Constructor types] \code{xs @ Cons \{ head: t; tail: u \}}
      (special-case: \code{t} is a singleton, we write
      \code{xs @ Cons \{ head = …; tail = … \}})
    \item[Decomposition] via \green{unfolding} (named fields),
      \green{refinement} (matching) and \green{folding} (subtyping)
    \item[Several possible types] \code{x @ (int, int)},\\
      \code{x @ ∃(y,z: value).\\~~(=y | y @ int, =z | z @ int)},
      \\\code{x @ ∃t.t}, etc.
  \end{description}
\end{frame}

\begin{frame}{\ldots that still isn't quite a logic}
  \mezzo remains a type system.
  \begin{itemize}
    \item far less connectives and rules
    \item $\tyatomic f{\tyarrow tu} \ast \tyatomic xt \not\leq
      \tyexists y\kterm {\tyatomic yu}$ (no implicitly callable ghost functions)
    \item no built-in disjunction (only tagged sums)
  \end{itemize}

  \bigskip

  \mezzo's type system feels like a limited fragment of \green{intuitionistic
  logic}.
\end{frame}

\begin{frame}{Subtraction: an unusual algorithm}
  \begin{itemize}
    \item Subtyping needs to be decided for \green{function calls} and for
      \green{function bodies}.
    \item %Subsumption rules allow instantiating quantifiers \green{implicitly}:
      Blurs the frontier between type-checking and logics.
    \item %Hence,
      The subtyping algorithm \emph{has to} \green{perform
      inference}%: introducing existentials in the goal, introducing universals
      %in the hypothesis (current permission).
  \end{itemize}
\end{frame}

\begin{frame}{More about subtraction}
  The operation is written $P - Q = R$ and assumes $P$ has been
  \emph{normalized} (all left-invertible rules have been applied).

  \bigskip
  \bigskip
  \bigskip


  {\Large
  \[
    \tikz[remember picture, baseline=-0.65ex] \node(sub-v) {$\mathcal{V}$};
    (
    \tikz[remember picture, baseline=-0.65ex] \node(sub-p) {P};
    -
    \tikz[remember picture, baseline=-0.65ex] \node(sub-q) {Q};
    ) =
    \tikz[remember picture, baseline=-0.65ex] \node(sub-vp) {$\mathcal{V'}$};
    \tikz[remember picture, baseline=-0.65ex] \node(sub-r) {R};
  \]
  }
  \begin{tikzpicture}[
    remember picture,
    overlay
  ]
    % V
    \node[
      draw, ultra thick, aluminium5, ellipse, fill=aluminium1
    ] (label-v) at ($(sub-v)+(-3em,-2em)$) {Binders};
    \draw[
      ultra thick, aluminium5, ->
    ] (label-v.north) to [out=90,in=180] (sub-v.west);
    \draw[
      ultra thick, aluminium5, ->
    ] (label-v.east) to [out=0,in=270] (sub-vp.south);

    % P
    \node[
      draw, ultra thick, aluminium5, ellipse, fill=aluminium1
    ] (label-p) at ($(sub-p)+(-3em,3em)$) {Hypothesis};
    \draw[
      ultra thick, aluminium5, ->
    ] (label-p.south) to [out=270,in=90] (sub-p.north);

    % Q
    \node[
      draw, ultra thick, aluminium5, ellipse, fill=aluminium1
    ] (label-q) at ($(sub-q)+(3em,3em)$) {Goal};
    \draw[
      ultra thick, aluminium5, ->
    ] (label-q.south) to [out=270,in=90] (sub-q.north);

    % R
    \node[
      draw, ultra thick, aluminium5, ellipse, fill=aluminium1
    ] (label-r) at ($(sub-r)+(3em,-2em)$) {Remainder};
    \draw[
      ultra thick, aluminium5, ->
    ] (label-r.north) to [out=90,in=0] (sub-r.east);

  \end{tikzpicture}

  \bigskip
  \bigskip

  This means:\\
  \qquad``with the instantiation choices from $\V'$, we get $P \leq Q \ast R$''.

\end{frame}

\begin{frame}{Subtraction example (1)}
  In order to type-check the application $f\ x$, we ask:
  \[
    P - \tyatomic f{\tyarrow {\alpha}{\beta}} \ast
        \tyatomic x{\alpha}\quad=\quad?
  \]

  \bigskip

  The algorithm must guess $\alpha$, $\beta$, and the remainder.
\end{frame}

\begin{frame}{Subtraction example (2)}
  \[
  \begin{array}{ll}
  & \tyatomic \ell{\tyconcrete \kcons{ \khead = h; \ktail = t }} \ast \\
  &\tyatomic h{\tyref\tyint} \ast \tyatomic t{\tylist {(\tyref\tyint)}} \\
  - \quad \\
  &\tyatomic \ell{\tylist{(\tyref\tyint)}} \\
  =  \quad\\
  &\tyatomic \ell{\tyconcrete \kcons{ \khead = h; \ktail = t }}
\end{array}
  \]
\end{frame}

\begin{frame}{Subtraction example (3)}
 \[
  \begin{array}{ll}
  & \tyatomic z{\exists\alpha, \forall\beta. (\alpha, \beta)} \\
  - \quad \\
  & \tyatomic z{\exists\alpha', \forall\beta'. (\alpha', \beta')} \\
  = \quad \\
  & ?
  \end{array}
\]
\end{frame}

\begin{frame}{Backtracking}
  Inference uses \emph{flexible} variables.

  \bigskip

  There may be \green{several solutions}:
  \[
  \tyatomic x\tyint - \tyatomic x\alpha =
  \tyatomic x\tyint\quad\text{ with }\quad
  \begin{cases}
    \alpha = \tyint \\
    \alpha = \tysingleton x \\
    \alpha = \top
  \end{cases}
  \]
  Not all solutions are explored: $\alpha$ could be $\tybar\beta p$.

  \bigskip

  There are many other backtracking points: right-introduction \emph{vs.}
  left-introduction, which atomic permission to focus\ldots
\end{frame}

\begin{frame}[fragile]{How to implement all that?}
  A module that takes care of running the proof search and providing an answer
  to:
  \[
  P-Q\quad = \quad?
  \]

  \bigskip

  \begin{minted}[fontsize=\footnotesize]{ocaml}
| TyQ (Forall, binding1, _, t'1), TyQ (Exists, binding2, _, t'2) ->
    par env judgement "Intro-Flex" [
      try_proof_root "Forall-L" begin
        let env, t'1, _ = bind_flexible_in_type env binding1 t'1 in
        sub_type env t'1 t2 >>=
        qed
      end;
      try_proof_root "Exists-R" begin
        let env, t'2, _ = bind_flexible_in_type env binding2 t'2 in
        sub_type env t1 t'2 >>=
        qed
      end
    ]
  \end{minted}
\end{frame}

\begin{frame}[plain]
  \backupslide
  \MzTitle{The proof search monad}
\end{frame}

\begin{frame}{}
  \begin{itemize}
    \item \mezzo is \green{not decidable}; however
    \item we \emph{know} \green{where} we want to backtrack;
      {\tiny(and where we do not want to go, e.g. $\pi \ast \pi' \ast \pi'' \leq p \ast q \ast r$)}
    \item we \emph{know} \green{which} branch is most likely to succeed;
      {\tiny($\alpha = \tyint$ usually better than $\alpha = \top$)}
    \item we \emph{know} that sub-branches \emph{terminate}
      {\tiny(no need to interleave)}
  \end{itemize}
\end{frame}

\begin{frame}{}
  In that case, adopt the library style where:
  \begin{itemize}
    \item the proof search algorithm looks \green{like the paper rules};
    \item you get a \green{derivation for free};
    \item the library works for \green{any logic} {\tiny(not in \mezzo, hence this
      paper)}
  \end{itemize}
\end{frame}

\begin{frame}[plain]
  \backupslide
  \MzTitle{The library, dissected}
\end{frame}

\begin{frame}[fragile]{From the library's perspective (1)}
  The client's logic must satisfy \code{LOGIC}.

  \bigskip

  \begin{minted}{ocaml}
module type LOGIC = sig
  type formula
  type rule_name
  type state
end
  \end{minted}
\end{frame}

\begin{frame}[fragile]{From the library's perspective (2)}
  The library defines derivation trees for \code{LOGIC}.

  \bigskip

  \begin{minted}{ocaml}
module Derivations = functor (L: LOGIC) -> struct
  type derivation = goal * rule
  and goal = L.state * L.formula
  and rule = L.rule_name * premises
  and premises = Premises of derivation list
end
  \end{minted}
\end{frame}

\begin{frame}[fragile]{From the library's perspective (3)}
  An \code{'a m} is the working \green{state} of a rule application.

  \vfill

  \begin{minted}[fontsize=\footnotesize]{ocaml}
module Make(Logic: LOGIC)(M: MONAD) = struct
  module Proofs = Derivations(Logic)
  include Proofs

  module L: MONOID with type t = Proofs.derivation list = struct
    type t = Proofs.derivation list
    let empty = []
    let append = List.append
  end

  include WriterT(M)(L) (*
    type 'a m = (L.t * 'a) M.m
    val return : 'a -> 'a m
    val tell : L.a -> unit m
    val ( >>= ) : 'a m -> ('a -> 'b m) -> 'b m
  *)

  ...
  \end{minted}
\end{frame}

\begin{frame}[fragile]{From the library's perspective (4)}
  An \code{'a outcome} is the \green{result} of a rule application.

  \bigskip

  \begin{minted}[fontsize=\footnotesize]{ocaml}
  ...

  type 'a outcome = ('a * derivation) M.m

  (* _Record_ a proof in the premises. *)
  val premise: 'a outcome -> 'a m
  (* _Conclude_ from the given premises. *)
  val prove:
    Logic.formula ->
    (Logic.state * rule_name) m ->
    Logic.state outcome
end
  \end{minted}
\end{frame}

\begin{frame}[fragile]{From the client's perspective (1)}
  \begin{minted}[fontsize=\footnotesize]{ocaml}

module MyLogic = struct ... end
module MyMonad = ProofSearchMonad.Make(MyLogic)(MExplore)

let rec solve (state: state) (goal: formula): state outcome =
  match goal with
  ...
  | And (g1, g2) ->
      prove goal begin
        premise (solve state g1) >>= fun state ->
        premise (solve state g2) >>= fun state ->
        state, R_And
      end
  ...
  \end{minted}
\end{frame}

\begin{frame}[fragile]{Let's refine (1)}
  \begin{minted}[fontsize=\footnotesize]{ocaml}
let qed rule = fun state -> state, rule


let rec solve (state: state) (goal: formula): state outcome =
  match goal with
  ...
  | And (g1, g2) ->
      prove goal begin
        premise (solve state g1) >>= fun state ->
        premise (solve state g2) >>=
        qed R_And
      end
  ...
  \end{minted}
\end{frame}

\begin{frame}[fragile]{Let's refine (2)}
  \begin{minted}[fontsize=\footnotesize]{ocaml}
val choice :
  Logic.formula ->
  'a list -> ('a -> (Logic.state * rule_name) m) -> Logic.state outcome

let rec solve (state: state) (goal: formula): state outcome =
  match goal with
  ...
  | Or (g1, g2) ->
      choice goal [ R_OrL, g1; R_OrR, g2 ] (fun (rule, g) ->
        premise (solve state g) >>=
        qed rule
      )
  ...
  \end{minted}
\end{frame}

\begin{frame}{In conclusion}
  \begin{itemize}
    \item A powerful library for writing the proof search of \mezzo
    \item Generalizes if your problem fits the earlier description
    \item Proof derivations for free
    \item We used them mostly for debugging {\tiny(In \mezzo, a post-processing
      phase tries to extract \emph{relevant} parts of a \emph{failed}
      derivation.)}
  \end{itemize}
\end{frame}

\begin{frame}{Next}
  \begin{itemize}
    \item \green{Extend} the library to record \emph{failed} derivations
      {\tiny(\code{choice} now records \emph{all} the things we tried; \code{>>=}
      records up to the first \emph{failed} premise)}
    \item See if compatible with more complex exploration strategies.
  \end{itemize}
\end{frame}

\end{document}

\begin{frame}{A glance at the type-checking rules}
  General form: $K, P \vdash e: t$. (K = kinding environment)

  \bigskip

  \small
  \begin{mathpar}
    \inferrule[Sub]
    {K; P_2 \vdash e: t_1 \\\\ P_1 \leq P_2 \\ t_1 \leq t_2}
    {K; P_1 \vdash e: t_2}

    \inferrule[Frame]
    {K; P_1 \vdash e: t}
    {K; P_1\ast P_2 \vdash e: \tybar{t}{P_2}}

    \inferrule[Read]
    {t \text{ is duplicable} \\\\
     P \text{ is } \tyatomic{x}{
                     \tyconcrete{A}{\ldots; f: t; \ldots}}
    }
    {K; P \vdash x.f: \tybar{t}{P}}

    \inferrule[Tuple]
    {}
    {K \vdash (x_1, \ldots, x_n): (\tysingleton {x_1}, \ldots, \tysingleton{x_n})}

    \inferrule[Application]
    {}
    {K; \tyatomic{x_1}{t_2 \rightarrow t_1} \ast \tyatomic{x_2}{t_2} \vdash \eapply{x_1}{x_2}: t_1}
  \end{mathpar}
\end{frame}

\begin{frame}{A glance at the subsumption relation}
  \small
  \begin{mathpar}
  \inferrule[DecomposeTuple]
    {}
    {
    \begin{array}{r@{}l}
    & \tyatomic{y}{(\ldots,t,\ldots)} \cr
    \subsub
    \tyexists x\kterm{(
      \tystar
        {& \tyatomic{y}{(\ldots, \tysingleton{x}, \ldots)}}
        {\tyatomic{x}{t}}
      )}
    \end{array}
    }

  \inferrule[EqualsForEquals]
    {}
    {
      \tystar
      {(y_1 = y_2)}
      {[y_1/x]P}
      \subsub
      \tystar
      {(y_1 = y_2)}
      {[y_2/x]P}
    }

  \inferrule[EqualityReflexive]
    {} % si x : term
    {\tyempty \leq (x = x)}

  \inferrule[Fold]
    {\tyconcrete{A}{\vec{f}: \vec{t}} \text{ is an unfolding of } \tyapp{X}{\vec{T}}}
    {
      \tyatomic{x}{\tyconcrete{A}{\vec{f}: \vec{t}}}
      \leq
      \tyatomic{x}{\tyapp{X}{\vec{T}}}
    }
  \end{mathpar}
\end{frame}

\begin{frame}{A glance at the subsumption relation (2)}
  \small
  \begin{mathpar}
  \inferrule[ForallElim]
    {}
    {\tyforall X\kappa P \leq [T/X]P}

  \inferrule[CopyDup]
    {P\text{ is duplicable}}
    {C[t] \ast P \leq C[\tybar tP] \ast P}

  \inferrule[HideDuplicablePrecondition]
    {P \text{ is duplicable}}
    {
      (\tyatomic{x}{\tyarrow{\tybar{t_1}{P}}{t_2}})
      \ast P
      \leq
      \tyatomic{x}{\tyarrow{t_1}{t_2}}
    }

  \inferrule[ExistsIntro]
    {}
    {[T/X]P \leq \tyexists X\kappa P}

  \inferrule[CoArrow]
    {u_1 \leq t_1 \\ t_2 \leq u_2}
    {\tyatomic{x}{\tyarrow{t_1}{t_2}} \leq \tyatomic{x}{\tyarrow{u_1}{u_2}}}

  \inferrule[Unfold]
    {\tyconcrete{A}{\vec{f}: \vec{t}} \text{ is an unfolding of } \tyapp{X}{\vec{T}} \\\\
     \tyapp{X}{\vec{T}} \text{ has only one branch}}
    {
      \tyatomic{x}{\tyapp{X}{\vec{T}}}
      \leq
      \tyatomic{x}{\tyconcrete{A}{\vec{f}: \vec{t}}}
    }
  \end{mathpar}
\end{frame}

\begin{frame}{A suitable representation of permissions}
  \mezzo is a powerful language: the type-checker is complex, because of the
  interaction between:
  \begin{itemize}
    \item \green{duplicable} \emph{vs.} \green{non-duplicable} permissions,
    \item equivalent permissions: \\
      \code{\green{z @ (=x, =y)} * x @ ref int * y @ ref int ≡ \\
      z @ (ref int, ref int)},
    \item inference (of type application): \code{cons [?] (x, y)},
    \item subtyping: \\
      \code{[a] duplicable a => (ref a) -> a ≡ \\
      \empty[y: value] (ref (=y)) -> (=y)},
    \item the frame rule…
  \end{itemize}
\end{frame}

\begin{frame}{The prototype}
  Backtracking \green{stops} at the expression level: we keep \green{one
  solution} when type-checking an expression.

  \bigskip

  The implementation relies on:
  \begin{itemize}
    \item efficient (good complexity) and easy-to-use (persistent) \green{data
      structures} for inference with backtracking (union-find, levels)
    \item fine-tuned \green{heuristics} (prioritize more likely solutions first)
  \end{itemize}
  Both required significant effort.

%   \bigskip

%   \sidenote{(Some performance limitations because of
%   binders.)}
\end{frame}

\begin{frame}[fragile]{\mezzo: reason about ownership}
  Consider list concatenation.

  \bigskip

  \begin{mzcode}
val append: [a] (
  consumes xs: list a,
  consumes ys: list a
) -> (zs: list a)
  \end{mzcode}
\end{frame}

\begin{frame}[fragile]{\mezzo: reason about ownership}
  \green{Example}: \code{list (ref int)}.

  \bigskip

  \bigskip

  \MzStartBubbles
  \begin{mzcode}
...#\MzProgramPoint#
let zs = append (xs, ys) in
...#\MzProgramPoint#
  \end{mzcode}

  \MzStartBubbles
  \MzAutoBubble{6em}{3em}{4cm}{Before function call}{%
    xs @ list (ref int) * ys @ list (ref int)
  }
  \MzAutoBubble{6em}{-3em}{4cm}{After function call}{%
    zs @ list (ref int)
  }
\end{frame}

\begin{frame}[fragile]{\mezzo: reason about ownership}
  \green{Example}: \code{list int}.

  \bigskip

  \MzStartBubbles
  \begin{mzcode}
...#\MzProgramPoint#
let zs = append (xs, ys) in
...#\MzProgramPoint#
  \end{mzcode}
  \MzStartBubbles
  \MzAutoBubble{6em}{3em}{6cm}{Before function call}{%
    xs @ list int * ys @ list int
  }
  \MzAutoBubble{6em}{-3em}{6cm}{After function call}{%
    xs @ list int * ys @ list int *\\
    zs @ list int
  }

\end{frame}

\end{document}
