\documentclass{sigplanconf}

\usepackage{amsmath}
\usepackage{tabularx}
\usepackage{listings}
\lstset{
  language=caml,
  basicstyle=\ttfamily,
}
\def\li{\lstinline}


\begin{document}

\special{papersize=8.5in,11in}
\setlength{\pdfpageheight}{\paperheight}
\setlength{\pdfpagewidth}{\paperwidth}

\conferenceinfo{CONF 'yy}{Month d--d, 20yy, City, ST, Country}
\copyrightyear{20yy}
\copyrightdata{978-1-nnnn-nnnn-n/yy/mm}
\doi{nnnnnnn.nnnnnnn}

% Uncomment one of the following two, if you are not going for the
% traditional copyright transfer agreement.

%\exclusivelicense                % ACM gets exclusive license to publish,
                                  % you retain copyright

%\permissiontopublish             % ACM gets nonexclusive license to publish
                                  % (paid open-access papers,
                                  % short abstracts)

\titlebanner{banner above paper title}        % These are ignored unless
\preprintfooter{short description of paper}   % 'preprint' option specified.

\title{Function pearl: the proof search monad}
\subtitle{Subtitle Text, if any}

\authorinfo{Jonathan Protzenko}
           {Microsoft Research}
           {protz@microsoft.com}

\maketitle

\begin{abstract}
  We present the proof search monad: the combination of the option monad, the
  backtracking monad, and the sequencing monad. The proof search monad allows
  writing a proof search engine that looks like the actual derivation rules,
  while generating a proof tree at the same time.
\end{abstract}

\category{CR-number}{subcategory}{third-level}

% general terms are not compulsory anymore,
% you may leave them out
\terms
term1, term2

\keywords
keyword1, keyword2

\section{Conjunctions of equalities}

Just the regular option monad. Lazyness is already built-in (the second
computation is suspended).

In this first version, we treat the simple problem of conjunction of
equalities\dots written in an elegant style with a persistent union-find,
functional, the option monad, because we may fail.

Lazyness already built-in, evaluation stops right after the first failing
premise has been hit.

Goal: build a derivation tree at the same time. Requires: recording the
premises, in order to build an application of the rule.

\bibliographystyle{abbrvnat}
\bibliography{local}

\end{document}
