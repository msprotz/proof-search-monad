\documentclass{easychair}

\usepackage{amsmath}
\usepackage{amssymb}
\usepackage{stmaryrd}
\usepackage{amstext}
\usepackage{amsthm}
\usepackage{hyperref}

\usepackage{tabularx}
\usepackage{listings}
  \def\li{\lstinline}
  \lstset{
    language=caml,
    flexiblecolumns=false,
    showstringspaces=false,
    basicstyle=\ttfamily,
    framesep=5pt,
    numberstyle=\tiny,
    numbersep=5pt,
    escapeinside={(*}{*)},
    morekeywords={above,abstract,adopts,alias,and,as,assert,begin,below,builtin,consumes,data,do,downto,duplicable,dynamic,else,empty,end,exclusive,explain,fact,fail,flex,for,from,fun,give,if,in,let,match,mutable,open,pack,perm,preserving,rec,take,taking,then,to,type,unknown,val,value,while,with,witness},
    deletekeywords={value},
  }
\usepackage{minted}
  \newminted[ocaml]{ocaml}{mathescape,fontsize=\small}
\usepackage{mathpartir}
  % Improvements to Didier's mathpartir package: make all rule names and
  % references use textsc, and also make references to rules done using the \Rule
  % command clickable in a PDF.
  % \let\TirName\textsc
  % \renewcommand{\RefTirName}[1]{\hypertarget{#1}{\TirName {#1}}}
  % \renewcommand{\DefTirName}[1]{\hyperlink{#1}{\TirName {#1}}}
  % \let\DefRule\RefTirName
  % \let\Rule\DefTirName
  \let\Rule\textsc


\newcommand{\fref}[1]{Figure~\ref{fig:#1}}
\newcommand{\sref}[1]{Section~\ref{sec:#1}}
\newcommand{\R}{\ensuremath{\mathcal{R}}} % Rigid
\newcommand{\f}[1]{\ensuremath{#1^?}} % Bold
\newcommand{\F}{\ensuremath{\mathcal{F}}} % Flexible
\newcommand{\V}{\ensuremath{\mathcal{V}}} % The prefix

\begin{document}

\title{Functional Pearl: the Proof Search Monad}
\titlerunning{The Proof Search Monad}

\author{Jonathan Protzenko}
\authorrunning{J. Protzenko}
\institute{
  Microsoft Research\\
  \email{protz@microsoft.com}}

\maketitle

\begin{abstract}
  We present the proof search monad, a set of combinators that allows one to
  write a proof search engine in a style that resembles closely the inference
  rules. The user calls functions such as \li+premise+, \li+prove+ or
  \li+choice+; the library then takes care of generating a derivation tree.
  Proof search engines written in this style enjoy: first, a one-to-one
  correspondence between the implementation and the theoretical rules, which
  makes manual inspection easier; second, proof witnesses ``for free'', which
  makes a verified, independent validation approach easier too.
\end{abstract}

\section{Theory and practice}

\subsection{A minimal theory}

We are concerned with proving the validity of logical formulas; that is, with
writing a search procedure that determines whether a given goal is satisfiable.
To get started, we consider a system made up of conjunctions of equalities,
along with existential quantifiers. Any free variables are assumed to be
universally quantified. For instance, one may want to prove the following
formula:

\begin{equation}
  \exists y.\ x = y
  \label{ex1}
\end{equation}

In order to show the validity of this judgement, one will build a proof
derivation using the rules from the logic, shown in \fref{logic} ($[x/y]P$ means
``substitute $x$ with $y$ in $P$''). For instance, proving Equation \ref{ex1}
requires applying \Rule{ExistsE}, then \Rule{Refl}.

\begin{figure}
  \centering
  \begin{mathpar}
    \inferrule[Refl]{
      \quad
    }{
      x = x
    }

    \inferrule[And]{
      P \\ Q
    }{
      P \wedge Q
    }

    \inferrule[ExistsE]{
      [x/y]P
    }{
      \exists y.\ P
    }
  \end{mathpar}
  \caption{A simple logic}
  \label{fig:logic}
\end{figure}

These rules embody the Truth of our logic, i.e. an omniscient reader may use
them to show with absolute certainty that a given formula is true.
%
However, if one wants to algorithmically determine whether a given formula is
true, \Rule{ExistsE} is useless. Indeed, unless the algorithm (solver) is
equipped with superpowers, it cannot magically guess, out of the blue, a
suitable $x$ in \Rule{ExistsE} that will ensure the remainder of the derivation
succeeds. To put it another way, $x$ is a free variable (a parameter) of
\Rule{ExistsE}; the whole point of writing a proof search algorithm is to 1)
find that \Rule{ExistsE} is the right rule to apply, and 2) find that $x$ is a
suitable value for instantiating $y$, because it will make $y = x$ succeed.

Hence, in order to build a \emph{search procedure} for that logic, one will
use another set of \emph{algorithmic} rules, which hopefully enjoy:
\begin{description}
  \item[soundness]: if the algorithmic rules succeed, then there exists a
    derivation in the logic that proves the validity of the original formula,
    and
  \item[completeness]: if the algorithmic rules fail, then there exists no
    derivation in the logic that would prove the validity of the original
    formula.
\end{description}

For instance, in our logic of existentially-quantified conjunctions of
equalities, one may want to use the algorithmic rules from
\fref{proof-system}. These rules differ from \fref{logic} in that they are
algorithmic; they take an input and return an output.

\begin{figure}
  \centering
  \begin{mathpar}
    \inferrule[Refl]{
      \quad
    }{
      V, \sigma \vdash x = x \dashv \sigma
    }

    \inferrule[Subst]{
      V, \sigma \vdash \sigma P \dashv \sigma'
    }{
      V, \sigma \vdash P \dashv \sigma'
    }

    \inferrule[Inst]{
      x \in V \\ \f y \in V \\ \f y \not\in \sigma \\\\
      V, \{ \f y \mapsto x \} \circ \sigma \vdash  P \dashv \sigma'
    }{
      V \vdash P \dashv \sigma'
    }

    \inferrule[And]{
      V, \sigma \vdash P \dashv \sigma' \\\\
      V, \sigma' \vdash Q \dashv \sigma''
    }{
      V \vdash P \wedge Q \dashv \sigma''
    }

    \inferrule[ExistsE]{
      V \uplus \f y, \sigma \vdash P \dashv \sigma'
    }{
      V, \sigma \vdash \exists y.\ P \dashv \sigma'_{|V}
    }
  \end{mathpar}
  \caption{Algorithmic proof rules}
  \label{fig:proof-system}
\end{figure}

In particular, in order to determine suitable values for the $x$ parameter in
\Rule{ExistsE}, the implementation reasons in terms of substitutions. $V$ is a
set of variables which may be substituted (recall that free variables are
considered universally quantified, hence not eligible for substitution);
variables that may be substituted are typeset as $\f y$.
The algorithm has internal state, that is, it carries a substitution $\sigma$.
Upon hitting an existential quantifier $\f y$, the algorithmic rules
\emph{open} $\f y$ and mark it as eligible for substitution. Later on (for
instance, upon hitting $\f y = x$), the algorithm may pick a
substitution for $\f y$ using \Rule{Inst}. A substitution may be applied at any
time (\Rule{Subst}). The preconditions of \Rule{Inst} guarantee that the
algorithm makes at most one choice for instantiating $\f y$.

In other words, the algorithmic rules \emph{defer} the \emph{instantiation} of
the existential quantifier until the shape of the proof obligation gives us a
\emph{hint} as to what exactly this instantiation should be. This
\emph{implementation technique} is known as \emph{flexible variables}.

The new algorithmic rules differ from the original logical rules significantly;
first, there are five rules for the algorithmic system, compared to just three
for the logical system. Second, these five rules do not map trivially to their
counterparts in the logical system. Third, these rules are only
algorithmic; the implementation that we are about to roll out uses an
optimized representation for substitutions (union-find), meaning that one not
only needs to check that the algorithmic rules are faithful to the logical
rules, but also that the implementation is faithful to the algorithmic rules.

This paper presents a library that allows one to write an implementation of the
algorithmic rules while automatically generating a derivation. The library
forces the client code to lay out premises, rule applications and
instantiations. The level of detail of the resulting derivation is left up to
the client code; they may wish to record the very compact rules from the logic,
or record more proof steps using the algorithmic rules. In any case, the
derivation serves as a proof witness; if the user wishes to do so, they can
write a validator that takes the witness and verifies that the derivation is,
indeed, correct.

The library has been used, in a preliminary form, to implement the core of the
Mezzo type-checker. This paper presents a cleaned-up, isolated version of this
library.

\subsection{An implementation with flexible variables and union-find}

The logic we present is a much simplified version of the logic (type system)
of Mezzo. In particular, this paper only mentions the right-exists quantifier;
Mezzo has all four possible combinations of left/right exists/forall. Generally,
in proof search, \emph{flexible variables} stem from the right-elimination of
existential quantifiers, or the left-elimination of universal quantifiers. The
right-elimination of universal quantifiers, or the left-elimination of
existential quantifiers give rise to universally-quantified variables, which are
called \emph{rigid}.

In order to simplify the problem, we assume that all existential variables have
been introduced as flexible variables already. That way, we won't be
sidetracked, talking about binders and the respective merits of De Bruijn
\emph{vs.} locally nameless. Furthermore, we assume that all instantiations of
flexible variables are legal. This is not true in general: for instance, if the
goal is $\forall x, \exists \f y, \forall z.\ P$, picking $\f y = z$ makes no
sense. Mezzo forbids this choice using \emph{levels}; in the present document,
we skip this discussion altogether and assume that ``all is well''. Finally,
although in a general setting, several rules may trigger for a given goal (this
is the case in Mezzo), the algorithmic set of rules we use is syntax-driven: the
syntactic shape of the goal determines which rule should be applied.

We thus restrict our formulas to conjunctions of equalities between variables.
The plan is to write a solver that takes,
as an input, a formula, and outputs a valid substitution, if any. That is,
write an algorithm that abides by the rules from \fref{proof-system}.
For instance, one may want to solve: $x = \f y \wedge z = z$.
A solution exists: the solver outputs $\sigma = \{ \f y \mapsto x \}$ as a valid
substitution that solves the input problem. However, if one attempts to solve:
$x = \f y \wedge \f y = z$, the solver fails to find a
proper substitution, and returns nothing. Indeed, the first clause demands that
$\f y$ substitutes to $x$, meaning that the second clause becomes $x = z$, which
always evaluates to false ($x$ and $z$ are two distinct rigid variables).

Once the algorithm has run, we obtain an output substitution $\sigma$. One can, if they
wish to do so, take the reflexive-transitive closure $\sigma^*$, and apply it to
a flexible variable (say, $\f y$) to recover the parameter of \Rule{ExistsE}
that should be used in the logical rules (here, $x$). This is all very informal
-- the point of the subsequent sections is to formalize the claim that the proof
search algorithm produces a proof witness.

\begin{figure}
  \centering
  \begin{ocaml}
type formula =                and descr =
| Equals of var * var           | Flexible
| And of formula * formula      | Rigid

and var = P.point             and state = descr P.state
  \end{ocaml}
  \caption{Formulas and state}
  \label{fig:formulas}
\end{figure}

We implement proof search in OCaml (\fref{formulas}); we do not use explicit
substitutions, but rather an optimized representation based on a union-find data
structure. The data type of formulas
is self-explanatory. Variables are implemented as equivalence classes in a
\emph{persistent} union-find data structure, which the module \li+P+ implements.
The $V, \sigma$ parameters in our rules are embodied by the \li+state+ type; just
like the $\sigma$ parameter is chained from one premise to another (\Rule{And}),
\li+state+ is an input and an output to the solver. Just like the $\sigma$ parameter
in the rules, a \li+state+ of the persistent union-find represents
a set of equations between variables. In a sense, \li+state+ is a specific
implementation of the theoretical $\sigma$ parameter. It represents a
substitution; in other words, this is what we want our solver to compute.

The choice of a union-find is irrelevant. All that matters is that we
pick a data structure that models substitutions, and that is \emph{persistent}.
Had we picked an explicit substitution instead of a union-find, the rest of the
discussion would have been the same.

\fref{solver} implements a solver for our minimal problem; written within the
\li+MOption+ monad, it returns either \li+Some state+ (in case a successful
substitution has been found), or \li+None+ if no solution exists. The solver is
complete.

\begin{figure}
  \centering
  \begin{ocaml}
module MOption = struct
  (* ... defines [return], [nothing] and [>>=] *)
end

let unify state v1 v2 =                         let rec solve state formula =
  match P.find v1 state, P.find v2 state with     match formula with
  | Flexible, Flexible                            | Equals (v1, v2) ->
  | Flexible, Rigid ->                                unify state v1 v2
      return (P.union v1 v2 state)                | And (f1, f2) ->
  | Rigid, Flexible ->                                solve state f1 >>= fun state ->
      return (P.union v2 v1 state)                    solve state f2
  | Rigid, Rigid ->
      if P.same v1 v2 state then
        return state
      else
        nothing
  \end{ocaml}
  \caption{Solver for the simplified problem}
  \label{fig:solver}
\end{figure}

The solver uses \li+MOption.>>=+ to sequence premises in the \li+And+ case. It
doesn't keep track of premises; it just ensures (thanks to \li+>>=+) that if the
first premise evaluates to \li+nothing+, the second premise is not evaluated,
since it is suspended behind a \li+fun+ expression (OCaml is a strict language).

\section{Building derivations}

There are two shortcomings with this solver. First, the \li+unify+ sub-routine
conflates several rules together. Indeed, the \li+return (P.union ...)+
expression hides a combination of \Rule{Inst} and \Rule{Refl}. Second, we
have no way to replay the proof to verify it independently. One may argue that
in this simplified example, the outputs substitution \emph{is} the proof
witness: one can just apply the substitution to the original
formula and verify that all the clauses are of the form $x = x$, without the
need for a proof tree. In the general case, however, the proof tree contains the
\Rule{ExistsE} rule, and proof witnesses are attached to arbitrary nodes of the
tree. We thus need to build a properly annotated proof tree in the general case.

\subsection{Defining proof trees}

One way to make the solver better is to make sure each step it performs
corresponds in an obvious manner to the application of an admissible rule. To
that effect, we define the data type of all three rules in our system, which we
apply to the functor of \emph{proof trees} (\fref{proof-trees}).

We record applications of \Rule{Inst}, \Rule{Refl} and \Rule{And}. This produces
a derivation tree that makes sure that the algorithm follows the algorithmic
rules from \fref{proof-system}. \sref{generate-logical} shows how to generate a
slightly different tree that matches the rules from \fref{logic}.

\begin{figure}
  \centering
\begin{ocaml}
(* These two modules belong to the library. *)
module type LOGIC = sig     module MakeProofTree (L: LOGIC) = struct
  type formula                type derivation = L.formula * rule
  type rule_name              and rule = L.rule_name * premises
end                           and premises = Premises of derivation list
                            end 

(* This is the client code using modules from the library. *)
module MyLogic = struct
  type formula = ... (* as before *)
  type rule_name = R_And | R_Refl | R_Inst
end
module MyProofTree = MakeProofTree(MyLogic)
\end{ocaml}
  \caption{The functor of proof trees (library and client code)}
  \label{fig:proof-trees}
\end{figure}

A \li+derivation+ tree is a pair of a \li+formula+ (the goal we wish the prove)
and a \li+rule+ (that we apply in order to prove the goal). A \li+rule+ has a name
and \li+premises+; the \li+premises+ type is simply a \li+derivation list+ (the
\li+Premises+ constructor is here to prevent a non-constructive type
abbreviation). When using the library, the client is expected to make sure
that each \li+rule_name+ is paired with the proper
number of premises (0 for \Rule{Refl}, 1 for
\Rule{Inst} and 2 for \Rule{And}); this is not enforced by the type system.

In the (simplified) sketch from \fref{proof-trees}, rule names are just constant
constructors, since the rule parameters (such as $x$ and $\f y$ in \Rule{Inst})
can be recovered from the \li+formula+. In the general case, the various
constructors of \li+rule_name+ do have parameters that record how one specific
rule was instantiated.

\subsection{Proof tree combinators}

We previously used the \li+>>=+ operator from the \li+MOption+ monad in order to
chain premises (\fref{solver}). We now need a new operator, that not only
\emph{binds} the result (i.e. stops evaluating premises after a failure, as
before), but also \emph{records} the premises in sequence, in order to build a
proper derivation. The former is still faithfully implemented by the option
monad; the latter is implemented by the writer monad.

Computations in the writer monad return a result (of type \li+'a+) along with a log of
elements (of type \li+L.a+). The (usual) \li+>>=+ and \li+return+ combinators operate on
the result part of the computation, while the (new) \li+tell+ combinator
operates on the logging part of the computation. This \li+tell+ combinator
appends a new element to the log. Appending elements to the log is done by way of the \li+MONOID+
module type, which essentially demands a value for the empty log, and a function to
append new entries into the log.

In order to get a new \li+>>=+ operator that combines the features of
the option and writer monads, we apply the \li+WriterT+ monad transformer to the
\li+MOption+ monad (\fref{writer}) and obtain \li+MWriter+, a monad whose
computations represent a sequence of derivations (the premises we have proved so far)
along with a result (the \li+state+ that we chain through the premises). These
computations are wrapped in \li+MOption.m+, that is, are wrapped within an
\li+option+ to account for a possible proof failure.

\begin{figure}
  \centering
\begin{ocaml}
module WriterT (M: MONAD) (L: MONOID): sig      module L = struct
  type 'a m = (L.a * 'a) M.m                      type a = MyProofTree.derivation list
  val return: 'a -> 'a m                          let empty = []
  val ( >>= ): 'a m -> ('a -> 'b m) -> 'b m       let append = List.append
                                                end
  val tell: L.a -> unit m
end = ...

module M = MOption
module MWriter = WriterT(M)(L)
\end{ocaml}
  \caption{The writer monad transformer (library code)}
  \label{fig:writer}
\end{figure}

A computation within this new monad has type (simplified after functor applications)
\li+(derivation list * state) option+. It represents a given point in the proof;
the solver is focused on a given rule, has reached a certain state, after
proving a certain list of premises.

We provide a convenience \li+qed+ combinator: once one has obtained the final
state, it pairs the state with the name of the rule we want to conclude with. It
makes the implementation of \li+solve+ (\fref{solver2}) more elegant.

Once all the premises have been proven, one needs to draw a horizontal line and
reach the conclusion of the proof. That is, take the final state and the list of
premises, and generate a \li+derivation+ that stands for the application of the
entire rule.

Contrary to the first implementation (\fref{solver}), where the working state
and the return value of \li+solve+ both had type \li+state option+, we now
distinguish between an \li+outcome+ (the result of a call to \li+solve+) and a
working state (a computation in the monad).

An outcome is, as we mentioned earlier, the pair of a final \li+state+ along
with a \li+derivation+ that justifies that we reached this state. The pair is
wrapped in \li+M.t+ (here, \li+option+): proving a formula may fail.

The type \li+outcome+ (\fref{combinators}) is parametric: it works for any
state that the client code uses. In other words, our library is generic with
regards to the particular \li+state+ type the client uses.

\begin{figure}
  \centering
\begin{ocaml}
(* This snippet is in the [MWriter(M)(L)] monad. Upon a first reading, think
   [module M = MOption]. *)
type 'a outcome = ('a * derivation) M.m

let premise (outcome: 'a outcome): 'a m =
  M.bind outcome (fun (state, derivation) ->
    tell [ derivation ] >>= fun () ->
    return state
  )

let prove (goal: goal) (x: ('a * rule_name) m): 'a outcome =
  M.(x >>= fun (premises, (state, rule)) ->
    return (state, (goal, (rule, Premises premises))))

let axiom (state: 'a) (goal: goal) (axiom: rule_name): 'a outcome =
  prove goal (return (state, axiom))

let qed r e =
  return (e, r)

let fail: 'a outcome =
  M.nothing
\end{ocaml}
  \caption{The high-level combinators for building proof derivations (library
  code)}
  \label{fig:combinators}
\end{figure}

We now have a duality between the \li+outcome+ type (the result of solving a
sub-goal) and the \li+m+ type (a computation within the monad, i.e. a working
state between two premises). Therefore, we introduce two high-level combinators:
\li+premise+ and \li+prove+. The former goes from \li+outcome+ to \li+m+: it
injects a new sub-goal as a premise of the rule we are currently trying to
prove. The latter goes from \li+m+ to \li+outcome+: if all premises have been
satisfied, it draws the horizontal line that builds a new node in the derivation
tree.

\begin{itemize}

  \item \li+premise+ is the composition of \li+tell+, which records the
    derivation for this sub-goal, and \li+return+, which passes the state on to
    the next sub-goal.

  \item \li+prove+ is a computation in the \li+M+ monad (here, \li+MOption+). If
    all the premises have been satisfied, it bundles them as a new node of the
    derivation tree. If a premise failed, then \li+x+ is \li+M.nothing+, and
    \li+prove+ also returns a failed outcome.

  \item \li+axiom+ is short-hand for a rule that requires no premises.

  \item \li+fail+ is for situations where no rule applies: this is a failed outcome.
\end{itemize}

\subsection{A solver in the new style}

\fref{solver2} demonstrates an implementation of \li+solve+ in the new style.
Compared to the previous implementation (\fref{solver}):
\begin{itemize}
  \item \li+prove_equality+ makes it explicit which rules are applied, and
    singles out two distinct rule applications in the flexible-rigid case;
  \item the premises of each rule are clearly identified;
  \item axioms and failure conditions are explicit,
  \item the \li+And+ case is easy to review manually, to make sure that no
    premise was forgotten.
\end{itemize}
This is, as mentioned previously, a minimal example that showcases the usage of
the library. In the implementation of Mezzo, switching the core of the
type-checker to this style revealed several bugs where premises were not
properly chained or simply forgotten.

\begin{figure}
  \centering
\begin{ocaml}
let rec prove_equality (state: state) (goal: formula) (v1: var) (v2: var) =
  let open MOption in
  match P.find v1 state, P.find v2 state with
  | Flexible, Flexible
  | Flexible, Rigid ->
      let state = P.union v1 v2 state in
      prove goal begin
        premise (prove_equality state goal v1 v2) >>=
        qed R_Instantiate
      end
  (* ... *)
  | Rigid, Rigid ->
      if P.same v1 v2 state then
        axiom state goal R_Refl
      else
        fail

let rec solve (state: state) (goal: formula): state outcome =
  match goal with
  | Equals (v1, v2) ->
      prove_equality state goal v1 v2
  | And (g1, g2) ->
      prove goal begin
        premise (solve state g1) >>= fun state ->
        premise (solve state g2) >>=
        qed R_And
      end
\end{ocaml}
  \caption{A solver written using the high-level combinators (client code)}
  \label{fig:solver2}
\end{figure}

\section{Backtracking}

\subsection{Limitations of the option monad}

We now extend our formulas with disjunctions (\fref{choice}). A consequence
is that we now need our base monad \li+M+ to offer a new operation; namely, one
that, among several possible choices, picks the first one that is not a failure.
We thus augment \li+MOption+ with a search combinator (\fref{choice}), which in
turn allows one to implement a high-level \li+choice+ combinator for our
library. The \li+choice+ combinator attempts to prove a \li+goal+ by trying a
function \li+f+ on several arguments \li+a+, each of which
is associated to a given rule. We can add one more branch to the \li+solve+
function, which attempts to prove a disjunction by first trying a
left-elimination (\Rule{Or-L}, \fref{proof-system2}), then a right-elimination
(\Rule{Or-R}).

\begin{figure}
  \centering
  \begin{mathpar}
    \inferrule[Or-L]{
      V \vdash P \dashv V'
    }{
      V \vdash P \vee Q \dashv V'
    }

    \inferrule[Or-R]{
      V \vdash Q \dashv V'
    }{
      V \vdash P \vee Q \dashv V'
    }
  \end{mathpar}
  \caption{New proof rules for disjunction}
  \label{fig:proof-system2}
\end{figure}

\begin{figure}
  \centering
\begin{ocaml}
(* We extend formulas with disjunctions. *)
type formula =
  (* ... *)
  | Or of formula * formula

(* The logic is also extended with two rules. *)
type rule_name =
  (* ... *)
  | R_OrL
  | R_OrR

module MOption = struct
  (* ... *)
  let rec search f = function
    | [] -> None
    | x :: xs ->
        match f x with
        | Some x -> Some x
        | None -> search f xs
end

(* Equipped with [search], we define the [choice] library combinator... *)
let choice (goal: goal) (args: 'a list) (f: 'a -> ('b * rule_name) m): 'b outcome =
  M.search (fun x -> prove goal (f x)) args

(* ...which one uses as follows: *)
let rec solve (state: state) (goal: formula): state outcome =
  match goal with
  (* ... *)
  | Or (g1, g2) ->
      choice goal [ R_OrL, g1; R_OrR, g2 ] (fun (r, g) ->
        premise (solve state g) >>=
        qed r
      )
\end{ocaml}
  \caption{The choice combinator (library and client code)}
  \label{fig:choice}
\end{figure}

The solver can now solve problems of the form $x = z \vee \f y = z$. It fails,
however, to solve problems of the form $(\f y = x \vee \f y = z) \wedge \f y =
z$. The reason is, the option monad is not powerful enough: upon finding a
suitable choice in the disjunction case, it commits to it and drops the other
one. In other words, when hitting the disjunction, \li+MOption+ commits to
$\sigma = \{ \f y \mapsto x \}$, instead of keeping $\sigma = \{ \f y \mapsto z
\}$ as a backup solution. Phrased yet again differently, we need to replace
\li+MOption+ with the non-determinism monad that will implement
\emph{backtracking}.

\subsection{The exploration monad}

Conceptually, we want to change our way of thinking; instead of thinking of
\li+solve+ as a function that returns \emph{a solution}, we now think of it as a
function that returns \emph{several possible solutions}. The state is now a
set of states, each of which represent a path in the search tree of derivation
trees.

The monad of non-determinism is implemented using lists; OCaml is a strict
language, so we write the non-determinism monad (also known as the exploration
or backtracking monad) using lazy lists (\fref{mexplore}).

\begin{figure}
  \centering
\begin{ocaml}
module LL = LazyList
module MExplore
  type 'a m = 'a LL.t
  let return = LL.one
  let ( >>= ) = LL.flattenl (LL.map f x)
  let nothing = LL.nil
  let search f l = LL.bind (LL.of_list l) f
end
\end{ocaml}
  \caption{The exploration monad}
  \label{fig:mexplore}
\end{figure}

The reader can now go back and replace \li+module M = MOption+ with
\li+module M = MExplore+ in \fref{writer}. The rest of the library remains
unchanged; the \li+solve+ function (the client code) is also unchanged; and the
combinators of the library now implement backtracking.

In particular, the earlier example of $(\f y = x \vee \f y = z) \wedge \f y = z$ is now successfully solved by the
library. Thanks to laziness, no extra computations occur; further solutions down
the lazy list are only evaluated if the first ones failed.

\section{Generating a proof tree using the logical rules}
\label{sec:generate-logical}

\section{Conclusion}

We presented a support library for writing a proof search engine using backtracking in
any given logic; indeed, the library is parameterized by: the type of formulas; the type of
rule applications; the internal state type of the client. By merely using the
combinators of the library, the client gets derivations built for free; this allows
a separate verifier to independently check the steps required to prove the
formula. By opting into the library, the client gets to rewrite their code in a
new syntactic style that makes rule application explicits, forbids ``bundled''
applications of multiple rules at the same time and clearly lays out the
premises required to prove a judgement. Since the code resembles the logical
rules, mistakes are easier to spot.

The logic presented in this paper is as simple as it gets. It does, however,
highlight the main concepts. A version of this library is used in the core of
Mezzo's type-checker. The version of the library used in Mezzo also builds
failed derivations; these failed derivations stop at the first failed premise
or, in case of a choice, list all the failed attempts. We have not yet explained
this last feature as a clean combination of monads and operators, but hope to do
so in the near future.

\bibliographystyle{abbrvnat}
\bibliography{local}

\end{document}
